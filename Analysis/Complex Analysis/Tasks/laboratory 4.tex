
\documentclass[12pt]{book}

\usepackage{matex}

\begin{document}

\block{ $ Martin Azpillaga $ }
{

\unit{ $ nth root determinations of a function $ }
{
	\baselineskip=30pt

	\thread{ $ Relationship between nth root determinations $ }
	{
		\letbe
		{
			X $ connected topological space $.
			\function{ f }{ X }{ \C \nonzero } $ continuous $.
			g,h $ nth root determinations of $ f.
		}
		\holds
		{
			\ex{ \zeta \in \mu_n(\C) }
			{
				h = \zeta g
			}
		}
		\demonstration
		{
			h,g $ continuous $, g \neq 0 \imp h/g $ continuous $.

			\all{ x \in X }
			{
				h(x)^n = f(x), \s g(x)^n = f(x).

				\left(\frac{ h(x) }{ g(x) }\right)^n = \frac{ h(x)^n }{ g(x)^n } =  \frac{ f(x) }{ f(x) } = 1.
				\frac{ h(x) }{ g(x) } \in \mu_n(\C).
			}.
			\im( h/g ) = \mu_n(\C) $ finite $.

			h/g $ constant over connected components $.

			X $ connected $ \imp h/g $ constant $.

			\ex{ \zeta \in \mu_n(\C) }
			{
				h = \zeta g
			}
		}
	}

	\newpage 
	\thread{ $ Cubic root determinations $ }
	{
		\letbe
		{
			h_0,h_1,h_2 $ cubic root determinations over $ \C \setminus (\R^- \times \{0\}) $ with $.
			h_0(1) = 1.
			h_1(1) = \exp( \frac{ 2\pi i }{ 3 } ).
			h_2(1) = \exp( \frac{ 4\pi i }{ 3 } ).
		}
		\study
		{
			\im( h_0 ), \im( h_1 ), \im( h_2 ).
			$Relationship with $ Log $ and $ Arg.
		}
		\demonstration
		{
			\all{ z \in \C \setminus (\R^- \times \{0\}) }
			{
				\sqrt[3]{ z } = \sqrt[3]{ \abs{ z } }\exp( \frac{ i(Arg(z) + 2k\pi) }{ 3 } ).

				\all{ k \in \dot{3} }
				{
					\sqrt[3]{ z } = \sqrt[3]{ \abs{ z } }\exp( \frac{ iArg(z) }{ 3 } ).
					arg(z) = \frac{ Arg(z) }{ 3 }.
					Arg(z) \in (-\pi,\pi) \imp arg(z) \in ( \frac{ -\pi }{ 3 }, \frac{ \pi }{ 3 }).
				}.

				\be{ \Omega_0 }{ \set{ z \in \C \with Arg(z) \in ( \frac{ -\pi }{ 3 }, \frac{ \pi }{ 3 }) } }.

				\all{ k \in \dot{3} + 1 }
				{
					\sqrt[3]{ z } = \sqrt[3]{ \abs{ z } }\exp( \frac{ i(Arg(z) + 2\pi)}{ 3 } ).
					arg(z) = \frac{ Arg(z) }{ 3 } + \frac{ 2\pi }{ 3 }.
					Arg(z) \in (-\pi,\pi) \imp arg(z) \in ( \frac{ \pi }{ 3 },  \pi  ).
				}.

				\be{ \Omega_1 }{ \set{ z \in \C \with Arg(z) \in ( \frac{ \pi }{ 3 },  \pi  ) } }.

				\all{ k \in \dot{3} + 2 }
				{
					\sqrt[3]{ z } = \sqrt[3]{ \abs{ z } }\exp( \frac{ iArg(z) }{ 3 } ).
					arg(z) = \frac{ Arg(z) }{ 3 } + \frac{ 4\pi }{ 3 }.
					Arg(z) \in (-\pi,\pi) \imp arg(z) \in ( \pi, \frac{ 5\pi }{ 3 } ).
				}

				\be{ \Omega_2 }{ \set{ z \in \C \with Arg(z) \in ( \pi, \frac{ 5\pi }{ 3 } ) } }.

				h_0(1) = 1 \imp \im( h_0 ) = \Omega_0.
				h_1(1) = \exp( \frac{ 2\pi i }{ 3 } ) \imp \im( h_1 ) = \Omega_1.
				h_2(1) = \exp( \frac{ 4\pi i }{ 3 } ) \imp \im( h_2 ) = \Omega_2.
			}.

			\particular
			{
				h_0(i) = \sqrt[3]{ \abs{ i } }\exp( \frac{ iArg(i) }{ 3 } )= \exp( \frac{ \pi }{ 6 } i ).
				h_0(i) = \sqrt[3]{ \abs{ i } }\exp( \frac{ iArg(i)+2\pi }{ 3 } )= \exp( \frac{ 5\pi }{ 6 } i ).
				h_0(i) = \sqrt[3]{ \abs{ i } }\exp( \frac{ i(Arg(i)+4\pi) }{ 3 } )= \exp( \frac{ 9\pi }{ 6 } i ).
			}
		}
	}
	
	\baselineskip=30pt
	


	
	
	

		
}}


\end{document}