
\documentclass[../Main/main]{subfiles}


\begin{document}

\renewcommand{\blockName}{ Martin Azpillaga }
\unit{ $ 1st laboratory $ }
{
	\proposition{ $ Existence of holomorphic functions $ }
	{
		\letbe
		{
			f \in \Hc( \D )
		}
		\study
		{
			\ex{ f \in \Hc( \D ) }
			{
				\all{ n \in \N }[ n \geq 2 ]
				{
					a) \; f( \pm \frac{ 1 }{ n } ) = \frac{ 1 }{ 2n + 1 }.
					b) \; f( \pm \frac{ 1 }{ n } ) = \frac{ 1 }{ n^2 }.
					c) \; |f( \frac{ 1 }{ n } ) |  = \frac{ 1 }{ \log( n+1 ) }.
					d) \; |f( \frac{ 1 }{ n } ) |  = \frac{ n }{ n+1 }
				}
			}
		}
		\demonstration
		{
			\step{ a) }
			{
				\be{ E_1 }{ \set{ +\frac{ 1 }{ n } + 0i \in \C \with n \in \N } }.
				\be{ E_2 }{ \set{ -\frac{ 1 }{ n } + 0i \in \C \with n \in \N } }.

				\limit{ \frac{ f(z) - f(0) }{ z - 0 } }{ E_1 } = \limit{ \frac{ f(\frac{ 1 }{ n }) }{ \frac{ 1 }{ n } } - \frac{ f(0) }{ \frac{ 1 }{ n } } }{ n } = \frac{ 1 }{ 2 } - \limit{\frac{ f(0) }{ \frac{ 1 }{ n } } }{ n }.
				\limit{ \frac{ f(z) - f(0) }{ z - 0 } }{ E_1 } \begin{cases} = \frac{ 1 }{ 2 } & f(0) = 0 \\ \nin \C & f(0) \neq 0 \end{cases}.
				\case{ f(0) = 0 }
				{
					\limit{ \frac{ f(z) - f(0) }{ z - 0 } }{ E_2 } = \limit{ \frac{ f(-\frac{ 1 }{ n }) }{ -\frac{ 1 }{ n } } }{ n } = - \frac{ 1 }{ 2 } \neq \limit{ \frac{ f(z) - f(0) }{ z - 0 } }{ E_1 }.
				}
				\nexists \; f \in \Hc(0) \suchthat f $ satisfies $ a).

				\particular{ \nexists \; f \in \Hc(\D) \suchthat f $ satisfies $ a) }
			}.


			\step{ b) }
			{
				\definedFunction{ f }{ \C }{ \C }{ z }{ z^2 }.

				\all{ n \in \N }[ n \geq 2 ]
				{
					f(\pm \frac{ 1 }{ n } ) = \frac{ 1 }{ n^2 }
				}.

				f $ satisfies $ b).


				\definedFunction{ \bar{f} }{ \R^2 }{ \R^2 }{ (x,y) }{ (u(x,y),v(x,y))=(x^2 - y^2, 2xy) }.

				\bar{f} \in $ Pol$( \R^2 ) \imp \bar{f} $ differentiable in $ \R^2.

				\all{ (x,y) \in \R^2 }
				{
					\partialderivative{ u(x,y) }{ x } = 2x = \partialderivative{ v(x,y) }{ y }.

					\partialderivative{ u(x,y) }{ y } = -2y=-\partialderivative{ v(x,y) }{ x }
				}.				

				f $ satisfies CR $.

				\conclude f \in \Hc( \R^2 ).

				\particular{ f \in \Hc( \D ) }
			}.

			\step{ c) }
			{
				$ Suppose $ \exists \; f \in \Hc(\D) \suchthat f $ satisfies $ c).
				f \in \Cc^0( \D ) \imp f(0) = f( \limit{ \frac{ 1 }{ n } }{ n } ) = \limit{ f( \frac{ 1 }{ n }) }{ n } = 0.

				\abs{ \limit{ \frac{ f(z) - f(0) }{ z - 0 } }{ E_1 } } = \limit{ \frac{ \abs{ f(\frac{ 1 }{ n }) } }{ \frac{ 1 }{ n } } }{ n } \nin \C.
				f \nin \Hc(0) $ absurd $.
			}.

			\step{ d) }
			{
				\definedFunction{ f }{ \C }{ \C }{ z }{ \frac{ 1 }{ z+1 } }.

				\all{ n \in \N }[ n \geq 2 ]
				{
					\abs{ f( \frac{ 1 }{ n } ) } = \frac{ 1 }{ \frac{ 1 }{ n } + 1 } = \frac{ n }{ n+1 }
				}.

				f $ satisfies $ d).


				\definedFunction{ \bar{f} }{ \R^2 }{ \R^2 }{ (x,y) }{ (u(x,y),v(x,y))=(\frac{ x+1 }{ (x+1)^2 + y^2 }, \frac{ -y }{ (x+1)^2 - y^2 }) }.

				\bar{f} \in $ Rat$( \R^2 ) \logicAnd \all{ (x,y) \in \R^2 }
				{
					(x+1)^2 + y^2 \neq 0
				}.

				\bar{f} $ differentiable in $ \R^2.

				\all{ (x,y) \in \R^2 }
				{
					\partialderivative{ u(x,y) }{ x } = \frac{ y^2 - (x+1)^2 }{ ((x+1)^2 + y^2)^2 } = \partialderivative{ v(x,y) }{ y }.

					\partialderivative{ u(x,y) }{ y } = \frac{ -2y(x+1) }{ ((x+1)^2 + y^2)^2 }=-\partialderivative{ v(x,y) }{ x }
				}.				

				f $ satisfies CR $.

				\conclude f \in \Hc( \R^2 ).

				\particular{ f \in \Hc( \D ) }
			}.
		}
	}


	\proposition{ $ Constant tests $ }
	{
		\letbe
		{
			\Omega \subset \C $ region $.
			f \in \Hc( \Omega )
		}
		\holds
		{
			f_{Re} = 0 \logicOr f_{Im} = 0 \imp f \in $ Cst$( \Omega ).

			\abs{ f } \in $ Cst$(\Omega) \imp f \in $ Cst$(\Omega).

			\im{ f } $ circumference $ \imp f \in \constant{ \Omega }

		}
		\demonstration
		{
			\step{ f_{Re} = 0 \logicOr f_{Im} = 0 }
			{
				\be{ u }{ f_{Re} }.
				\be{ v }{ f_{Im} }.

				f \in \Hc( \Omega ) \imp f $ satisfies CR in $ \Omega.

				\partialderivative{ u }{ x } =  \partialderivative{ v }{ y } = 0.
				\partialderivative{ u }{ y } = -\partialderivative{ v }{ x } = 0.

				\step{ $Null diferential test$ }
				{
					\Omega $ connex $ \imp u,v \in \constant{ \Omega }.
				}

				u,v \in \constant{ \Omega } \imp f \in \constant{ \Omega } 
			}.

			\step{ \abs{ f } \in $ Cst$(\Omega) }
			{

				\definedFunction{ \abs{f} }{ \R^2 }{ \R }{ (x,y) }{ \sqrt{ u(x,y)^2 + v(x,y)^2 } }.

				\abs{f} \in \constant{ \Omega } \imp \ex{ a \in \R }
				{
					\sqrt{ u(x,y)^2 + v(x,y)^2 } = a.

					u(x,y)^2 + v(x,y)^2 = a^2.

					2 \partialderivative{ u(x,y) }{ x } + 2 \partialderivative{ v(x,y) }{ x } = 0.
					
					2 \partialderivative{ u(x,y) }{ y } + 2 \partialderivative{ v(x,y) }{ y } = 0.

					f \in \Hc( \Omega ) \imp f $ satisfies CR in $ \Omega.

					2 \partialderivative{ v(x,y) }{ y } + 2 \partialderivative{ v(x,y) }{ x } = 0.
					
					-2 \partialderivative{ v(x,y) }{ x } + 2 \partialderivative{ v(x,y) }{ y } = 0.

					+: \; 4\partialderivative{ v(x,y) }{ y } = 0 \imp \partialderivative{ v(x,y) }{ y } = 0.

					-: \; 4\partialderivative{ v(x,y) }{ x } = 0 \imp \partialderivative{ v(x,y) }{ x } = 0.

					\step{ $Null differential test$ }
					{
						\Omega $ connex $ \imp u,v \in \constant{ \Omega }
					}.

					u,v \in \constant{ \Omega } \imp f \in \constant{ \Omega }

				}
			}.

			\step{ \im ( f ) $ circumference $ }
			{
				\ex{ (x_0,y_0) \in \R^2, r \in \R^+ }
				{
					\im( f ) = C_r(x_0,y_0).

					\definedFunction{ \bar{f} }{ \R^2 }{ C_r(x_0,y_0) }{ (x,y) }{ (r \cos( x-x_0) , r \sin( y-y_0 )) }.

					\all{ (x,y) \in \Omega }
					{
						\abs{ \bar{f} }(x,y) = \sqrt{ r^2( \cos^2( x-x_0 ) + \sin^2( y-y_0 ) } = r
					}.

					\abs{ f } \in \constant{ \Omega } \imp f \constant{ \Omega }

				}
			}
		}
	}
	
	

	\proposition{ $ Real part of holomorphic functions $ }
	{
		\letbe
		{
			\Omega \subset \R^2 $ region $.
			u \in \Cc^2(\Omega) \suchthat \ex{ f \in \Hc(\Omega) }
			{
				f_{Re} = u
			}
		}
		\showthat
		{
			\partialderivative{ u }{ xx } + \partialderivative{ u }{ yy } = 0
		}
		\study
		{
			\ex{ f \in \Hc(\Omega) }
			{
				a) \; f_{Re}(x,y) = x^2+y^2.
				b) \; f_{Re}(x,y) = x(x+1) -y^2.
				c) \; \all{ \alpha \in \R }
				{
					f_{Re} = y^3 + \alpha x^2y \logicAnd \Omega = \C
				}
			}
		}
		\demonstration
		{
			\step{ \partialderivative{ u }{ xx } + \partialderivative{ u }{ yy } = 0 }
			{
				\be{ u }{ f_{Re} }, \be{ v }{ f_{Im} }.

				f \in \Hc( \Omega ) \imp f $ satisfies CR in $ \Omega.

				\partialderivative{ u }{ x } = \partialderivative{ v }{ y } \imp \partialderivative{ u }{ xx } = \partialderivative{ v }{ xy }.

				\partialderivative{ u }{ y } = - \partialderivative{ v }{ x } \imp \partialderivative{ u }{ yy } = - \partialderivative{ v }{ xy }.

				\conclude \partialderivative{ u }{ xx } + \partialderivative{ u }{ yy } = 0
			}.

			\step{ f_{Re}(x,y) = x^2+y^2 }
			{

				\partialderivative{ u }{ xx } + \partialderivative{ u }{ yy } = 4 \neq 0.

				\nexists f \in \Hc(\Omega) \suchthat f_{Re}(x,y) = x^2 + y^2
			}.

			\step{ f_{Re}(x,y) = x(x+1) -y^2 }
			{
				\definedFunction{ \bar{f} }{ \Omega }{ \R^2 }{ (x,y) }{ (u(x,y),v(x,y))=(x(x+1) - y^2, 2xy + y) }.

				\bar{f} \in \polynomial{ \Omega } \imp \bar{f} $ differentiable in $ \Omega.

				\all{ (x,y) \in \Omega }
				{
					\partialderivative{ u(x,y) }{ x } = 2x+1 = \partialderivative{ v(x,y) }{ y }.

					\partialderivative{ u(x,y) }{ y } = -2y = -\partialderivative{ v(x,y) }{ x }.
				}

				f $ satisfies CR in $ \Omega.

				f \in \Hc(\Omega) \logicAnd f_{Re} = u
			}.

			\step{ f_{Re}(x,y) = y^3 + \alpha x^2y }
			{
				\step{ f $ has to satisfy CR in $ \C }
				{
					\all{ (x,y) \in \R^2 }
					{
						\partialderivative{ v(x,y) }{ y } = \partialderivative{ u(x,y) }{ x } = 2\alpha xy.

						\partialderivative{ v(x,y) }{ x } = -\partialderivative{ u(x,y) }{ y } = - 3y^2 - \alpha x^2.
						v(x,y) = \alpha xy^2 + c(x).
						v(x,y) = -3xy^2 - \frac{ \alpha }{ 3 }x^3 + c(y).
						
						\alpha = -3, \; c(x) = x^3, \; c(y) = 0.

						v(x,y) = -3xy^2 + x^3

					}
				}
			}

		}
	}
	
	
	
	
}



\end{document}