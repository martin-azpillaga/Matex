



\documentclass[../Main/main]{subfiles}


\begin{document}


\unit{ $ New $ }
{

	\proposition{ $ logarithm is holomorphic $ }
	{
		\letbe
		{
			\function{ log }{ \C \setminus e^{i\alpha}(-\infty,0] }{ B_\alpha }
		}
		\holds
		{
			log \in \Hc.
			log' = \frac{ 1 }{ z }
		}
		\demonstration
		{
			log $ continuous $.
			\exp( \log( z ) ) = z.
			\omega = \log( z ).
			\omega_0 = \log( z_0 ).
			\exp( \omega ) = z.
			\exp( \omega_0 ) = z_0.
			log $ continuous $ \imp w \convergesto w_0 $ if $ z = z_0.
			\frac{ log(z) - log(z_0) }{ z - z_0 } = \frac{ w- w_0 }{ e^w - e^{w_0} }.
			 = \limit{ \frac{ 1 }{ \frac{ e^w - e^{w_0} }{ w - w_0 } } }{ w }[ w_0 ] = \frac{ 1 }{ e^{w_0} } = \frac{ 1 }{ z_0 }
		}
	}
	
	
	\proposition{ $ Complex Powers $ }
	{
		\letbe
		{
			z^w = e^{wlog z} = e^{w(log|z| + iarg(z)}
		}
		\holds
		{
			z^w = \exp( n \log( z ) )
		}
		\demonstration
		{
			z = \exp( \log( z ) ).
			z^w = \exp( \log( z ) )^n = \product{ \exp( \log( z ) ) }{ i }[ 1 ][ n ] = \exp( n \log( z ) )
		}
	}
	
	
	\example{ $ i powers $ }
	{
		\letbe
		{
			i \in \C
		}
		\is{ i^i }{ $ an infinite value set $ }
		{
			\all{ k \in \Z }
			{
				i^i = \exp( i \log( i ) )=\exp( i^2(\frac{ \pi }{ 2 } + 2k\pi) ) = \exp( -(\frac{ \pi }{ 2 } + 2k\pi) )
			}
			
		}
	}
	
	
	\proposition{ $ Different values of powers $ }
	{
		\letbe
		{
			z,w \in \C
		}
		\holds
		{
			w \in \Z \imp z^w $ unique $.
			w = \frac{ p }{ q } \in \Q \imp z^w $ has q values $.
			w \in \R \setminus \Q \imp z^w $ has infinite values $ 
		}
	}
	
	
	\example{ $ Different logarithm definitions $ }
	{
		\letbe
		{
			\Omega = \C \setminus \set{ (x,y) \in \C \with x < 0, y = 0 }.
			\Omega_2 = \C \setminus \set{ (x,y) \in \C \with x > 0, y = x }.
			\Omega_3 = \C \setminus \set{ (x,y) \in \C \with x \in (-1,0), y = 0 } \cup \set{ (x,y) \in \C \with x=-1,y\in(0,\pi) } \cup \set{ (x,y= \in \C \with x > -1, y = \frac{ \pi }{ 2 } }
		}
		\study
		{
			arg(i),arg(2i) $ in function of $ arg(1)
		}	
		\start
		{
			arg(1) = 0 \imp arg(i) = \frac{ \pi }{ 2 }.
			arg(1) = -2\pi \imp arg(i) = \frac{ -3\pi }{ 2 }.
			arg(i) = \frac{ -3\pi }{ 2 } \imp arg(1) = -2\pi..

			
			arg(1) = 0 \imp arg(i) = \frac{ -3\pi }{ 2 }.
			arg(1) = -2\pi \imp arg(\pi) = \frac{ -7\pi }{ 2 }..

			arg(1) = 0 \imp arg(i) = \frac{ \pi }{ 2 }.
			arg(1) = 0 \imp arg(2i) =\frac{ -3\pi }{ 2 }.
			arg(i) = \frac{ -3\pi }{ 2 } \imp arg(2i) = \frac{ -7\pi }{ 2 }
		}
	}
	
	
	\definition{ $ Simple fraction form $ }
	{
		\letbe
		{
			f \in \rational( \C )
		}
		\name{ $ simple fraction form of $ f }
		{
			\summation{ \frac{ a_i }{ (x-a_i) } }{ i }[ 1 ][ r ] = f
		}
	}
	
	
	\definition{ $ argument wedge $ }
	{
		\letbe
		{
			\gamma $ differentiable curve $
		}
		\then{ \gamma }{ $ an argument curve $ }
		{
			\gamma(0) = 0.
			\gamma(1) = \infty
		}
	}
	
	
	log(f(z))?\\

	\definition{ $ log(f(x)) determination $ }
	{
		\letbe
		{
			(X,d) $ metric space $.
			\function{ f }{ X }{ \C \nonzero } $ continuous $.
			\function{ h }{ X }{ \C }
		}
		\then{ h }{ $ a log(f) determination $ }
		{
			h $ continuous $.
			\exp( h(x) ) = f(x).
			$ h selecciona uno de los posibles valores log(h(x) para cada x de manera continua en x $

		}
	}
	
	
	\definition{ $ argument determination $ }
	{
		\letbe
		{
			(X,d) $ metric space $.
			\function{ f }{ X }{ \C \nonzero } $ continuous $.
			\function{ a }{ X }{ \R }
		}
		\then{ a }{ $ an argument determination $ }
		{
			a $ continuous $.
			|f(x)|\exp( ia(x) ) = f(x)
		}
	}
	
	
	\proposition{ $ induced argument/logarithm determination $ }
	{
		\letbe
		{
			statements.
		}
		\holds
		{
			log(f(x) = ln|f(x)| + i arg(f(x)).
			h(x) = ln|f(x)| + ia(x)
		}
		\demonstration
		{
			demonstration.
		}
	}
	
	
	\definition{ $ principal logarithm $ }
	{
		\letbe
		{
			\function{ log }{ \C \setminus (-\infty,0) }{ \C }
		}
		\then{ log }{ $ principal logarithm $ }
		{
			log(1) = 0
		}
		\denote
		{
			property \as notation.
		}
	}
	
	
	\definition{ $ principal argument $ }
	{
		\letbe
		{
			\function{ Arg }{ \C }{ (-\pi,\pi) }
		}
		\then{ Arg }{ $ principal argument $ }
		{
			Log(z) = ln|z| + i Arg(z)
		}
	}
	
	
	\example{ $ identity logarithm determination $ }
	{
		\letbe
		{
			X = \C \nonzero.
			f(z) \as z
		}
		\holds
		{
			\nexists \; log $ logarithm determination over $ X
		}
	}
	
	
	\example{ $ exponential logarithm determination $ }
	{
		\letbe
		{
			X = [0,1].
			f(x) = \exp( 4\pi i x )
		}
		\is{ h(x) = 4i\pi x }{ $ a logarithm determination of $ f $ $ }
		{
			h $ continuous over $ X.
			\exp( 4\pi i x ) = 4\pi i x = f(x)
		}
	}
	
	
	\proposition{ $ logarithm determinations over image f $ }
	{
		\letbe
		{
			statements.
		}
		\holds
		{
			$exists det of log in Im f $\imp $ exists det log in X $.
			$exists det log in X $\suchthat $no exists det log in Im f$
		}
		\demonstration
		{
			demonstration.
		}
	}
	
	
	\proposition{ $ two determinations of log f $ }
	{
		\letbe
		{
			(X,d) $ connex metric space $.
			\function{ f }{ X }{ \C } $ continuous $.
			h_1,h_2 $ determination of logarithm of $ f
		}
		\holds
		{
			\ex{ k \in \N }
			{
				h_1 = h_2 + 2k\pi i
			}
			
		}
		\demonstration
		{
			\all{ x \in X }
			{
				k(x) = \frac{ h_1(x) - h_2(x) }{ 2\pi i } $ continuous $.
				k $integer function $.
				X $connex $ \imp k $ constant $
			}
			\be{ k }{ k(x) }
		}
	}
	
	
	\proposition{ $ curves have logarithm determination $ }
	{
		\letbe
		{
			\function{ \gamma }{ [a,b] }{ \C \nonzero } $ continuous curve $
		}
		\holds
		{
			\ex{ \function{ log }{ \gamma* }{ \C } }
			{
				log $ logaithm determination over $ \gamma*
			}
		}
		\demonstration
		{
			demonstration.
		}
	}
	
	
	
	
	
	
	
	

}


\end{document}