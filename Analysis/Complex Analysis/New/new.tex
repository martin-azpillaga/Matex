



\documentclass[../Main/main]{subfiles}


\begin{document}


\unit{ $ New $ }
{

	% \proposition{ $ logarithm is holomorphic $ }
	% {
	% 	\letbe
	% 	{
	% 		\function{ log }{ \C \setminus e^{i\alpha}(-\infty,0] }{ B_\alpha }
	% 	}
	% 	\holds
	% 	{
	% 		log \in \Hc.
	% 		log' = \frac{ 1 }{ z }
	% 	}
	% 	\demonstration
	% 	{
	% 		log $ continuous $.
	% 		\exp( \log( z ) ) = z.
	% 		\omega = \log( z ).
	% 		\omega_0 = \log( z_0 ).
	% 		\exp( \omega ) = z.
	% 		\exp( \omega_0 ) = z_0.
	% 		log $ continuous $ \imp w \convergesto w_0 $ if $ z = z_0.
	% 		\frac{ log(z) - log(z_0) }{ z - z_0 } = \frac{ w- w_0 }{ e^w - e^{w_0} }.
	% 		 = \limit{ \frac{ 1 }{ \frac{ e^w - e^{w_0} }{ w - w_0 } } }{ w }[ w_0 ] = \frac{ 1 }{ e^{w_0} } = \frac{ 1 }{ z_0 }
	% 	}
	% }
	
	
	% \proposition{ $ Complex Powers $ }
	% {
	% 	\letbe
	% 	{
	% 		z^w = e^{wlog z} = e^{w(log|z| + iarg(z)}
	% 	}
	% 	\holds
	% 	{
	% 		z^w = \exp( n \log( z ) )
	% 	}
	% 	\demonstration
	% 	{
	% 		z = \exp( \log( z ) ).
	% 		z^w = \exp( \log( z ) )^n = \product{ \exp( \log( z ) ) }{ i }[ 1 ][ n ] = \exp( n \log( z ) )
	% 	}
	% }
	
	
	% \example{ $ i powers $ }
	% {
	% 	\letbe
	% 	{
	% 		i \in \C
	% 	}
	% 	\is{ i^i }{ $ an infinite value set $ }
	% 	{
	% 		\all{ k \in \Z }
	% 		{
	% 			i^i = \exp( i \log( i ) )=\exp( i^2(\frac{ \pi }{ 2 } + 2k\pi) ) = \exp( -(\frac{ \pi }{ 2 } + 2k\pi) )
	% 		}
			
	% 	}
	% }
	
	
	% \proposition{ $ Different values of powers $ }
	% {
	% 	\letbe
	% 	{
	% 		z,w \in \C
	% 	}
	% 	\holds
	% 	{
	% 		w \in \Z \imp z^w $ unique $.
	% 		w = \frac{ p }{ q } \in \Q \imp z^w $ has q values $.
	% 		w \in \R \setminus \Q \imp z^w $ has infinite values $ 
	% 	}
	% }
	
	
	% \example{ $ Different logarithm definitions $ }
	% {
	% 	\letbe
	% 	{
	% 		\Omega = \C \setminus \set{ (x,y) \in \C \with x < 0, y = 0 }.
	% 		\Omega_2 = \C \setminus \set{ (x,y) \in \C \with x > 0, y = x }.
	% 		\Omega_3 = \C \setminus \set{ (x,y) \in \C \with x \in (-1,0), y = 0 } \cup \set{ (x,y) \in \C \with x=-1,y\in(0,\pi) } \cup \set{ (x,y= \in \C \with x > -1, y = \frac{ \pi }{ 2 } }
	% 	}
	% 	\study
	% 	{
	% 		arg(i),arg(2i) $ in function of $ arg(1)
	% 	}	
	% 	\start
	% 	{
	% 		arg(1) = 0 \imp arg(i) = \frac{ \pi }{ 2 }.
	% 		arg(1) = -2\pi \imp arg(i) = \frac{ -3\pi }{ 2 }.
	% 		arg(i) = \frac{ -3\pi }{ 2 } \imp arg(1) = -2\pi..

			
	% 		arg(1) = 0 \imp arg(i) = \frac{ -3\pi }{ 2 }.
	% 		arg(1) = -2\pi \imp arg(\pi) = \frac{ -7\pi }{ 2 }..

	% 		arg(1) = 0 \imp arg(i) = \frac{ \pi }{ 2 }.
	% 		arg(1) = 0 \imp arg(2i) =\frac{ -3\pi }{ 2 }.
	% 		arg(i) = \frac{ -3\pi }{ 2 } \imp arg(2i) = \frac{ -7\pi }{ 2 }
	% 	}
	% }
	
	
	% \definition{ $ Simple fraction form $ }
	% {
	% 	\letbe
	% 	{
	% 		f \in \rational( \C )
	% 	}
	% 	\name{ $ simple fraction form of $ f }
	% 	{
	% 		\summation{ \frac{ a_i }{ (x-a_i) } }{ i }[ 1 ][ r ] = f
	% 	}
	% }
	
	
	% \definition{ $ argument wedge $ }
	% {
	% 	\letbe
	% 	{
	% 		\gamma $ differentiable curve $
	% 	}
	% 	\then{ \gamma }{ $ an argument curve $ }
	% 	{
	% 		\gamma(0) = 0.
	% 		\gamma(1) = \infty
	% 	}
	% }
	
	
	% log(f(z))?\\

	% \definition{ $ log(f(x)) determination $ }
	% {
	% 	\letbe
	% 	{
	% 		(X,d) $ metric space $.
	% 		\function{ f }{ X }{ \C \nonzero } $ continuous $.
	% 		\function{ h }{ X }{ \C }
	% 	}
	% 	\then{ h }{ $ a log(f) determination $ }
	% 	{
	% 		h $ continuous $.
	% 		\exp( h(x) ) = f(x).
	% 		$ h selecciona uno de los posibles valores log(h(x) para cada x de manera continua en x $

	% 	}
	% }
	
	
	% \definition{ $ argument determination $ }
	% {
	% 	\letbe
	% 	{
	% 		(X,d) $ metric space $.
	% 		\function{ f }{ X }{ \C \nonzero } $ continuous $.
	% 		\function{ a }{ X }{ \R }
	% 	}
	% 	\then{ a }{ $ an argument determination $ }
	% 	{
	% 		a $ continuous $.
	% 		|f(x)|\exp( ia(x) ) = f(x)
	% 	}
	% }
	
	
	% \proposition{ $ induced argument/logarithm determination $ }
	% {
	% 	\letbe
	% 	{
	% 		statements.
	% 	}
	% 	\holds
	% 	{
	% 		log(f(x) = ln|f(x)| + i arg(f(x)).
	% 		h(x) = ln|f(x)| + ia(x)
	% 	}
	% 	\demonstration
	% 	{
	% 		demonstration.
	% 	}
	% }
	
	
	% \definition{ $ principal logarithm $ }
	% {
	% 	\letbe
	% 	{
	% 		\function{ log }{ \C \setminus (-\infty,0) }{ \C }
	% 	}
	% 	\then{ log }{ $ principal logarithm $ }
	% 	{
	% 		log(1) = 0
	% 	}
	% 	\denote
	% 	{
	% 		property \as notation.
	% 	}
	% }
	
	
	% \definition{ $ principal argument $ }
	% {
	% 	\letbe
	% 	{
	% 		\function{ Arg }{ \C }{ (-\pi,\pi) }
	% 	}
	% 	\then{ Arg }{ $ principal argument $ }
	% 	{
	% 		Log(z) = ln|z| + i Arg(z)
	% 	}
	% }
	
	
	% \example{ $ identity logarithm determination $ }
	% {
	% 	\letbe
	% 	{
	% 		X = \C \nonzero.
	% 		f(z) \as z
	% 	}
	% 	\holds
	% 	{
	% 		\nexists \; log $ logarithm determination over $ X
	% 	}
	% }
	
	
	% \example{ $ exponential logarithm determination $ }
	% {
	% 	\letbe
	% 	{
	% 		X = [0,1].
	% 		f(x) = \exp( 4\pi i x )
	% 	}
	% 	\is{ h(x) = 4i\pi x }{ $ a logarithm determination of $ f $ $ }
	% 	{
	% 		h $ continuous over $ X.
	% 		\exp( 4\pi i x ) = 4\pi i x = f(x)
	% 	}
	% }
	
	
	% \proposition{ $ logarithm determinations over image f $ }
	% {
	% 	\letbe
	% 	{
	% 		statements.
	% 	}
	% 	\holds
	% 	{
	% 		$exists det of log in Im f $\imp $ exists det log in X $.
	% 		$exists det log in X $\suchthat $no exists det log in Im f$
	% 	}
	% 	\demonstration
	% 	{
	% 		demonstration.
	% 	}
	% }
	
	
	% \proposition{ $ two determinations of log f $ }
	% {
	% 	\letbe
	% 	{
	% 		(X,d) $ connex metric space $.
	% 		\function{ f }{ X }{ \C } $ continuous $.
	% 		h_1,h_2 $ determination of logarithm of $ f
	% 	}
	% 	\holds
	% 	{
	% 		\ex{ k \in \N }
	% 		{
	% 			h_1 = h_2 + 2k\pi i
	% 		}
			
	% 	}
	% 	\demonstration
	% 	{
	% 		\all{ x \in X }
	% 		{
	% 			k(x) = \frac{ h_1(x) - h_2(x) }{ 2\pi i } $ continuous $.
	% 			k $integer function $.
	% 			X $connex $ \imp k $ constant $
	% 		}
	% 		\be{ k }{ k(x) }
	% 	}
	% }
	
	
	% \proposition{ $ curves have logarithm determination $ }
	% {
	% 	\letbe
	% 	{
	% 		\function{ \gamma }{ [a,b] }{ \C \nonzero } $ continuous curve $
	% 	}
	% 	\holds
	% 	{
	% 		\ex{ \function{ log }{ \gamma* }{ \C } }
	% 		{
	% 			log $ logaithm determination over $ \gamma*
	% 		}
	% 	}
	% 	\demonstration
	% 	{
	% 		$graphical demonstration $
	% 	}
	% }
	
	
	% \example{ $ logarithm determination of curves $ }
	% {
	% 	\letbe
	% 	{
	% 		\function{ \gamma }{ [0,1] }{ \C }.
	% 		\gamma(0) = (1,0).
	% 		\gamma(\frac{ 1 }{ 3 }) = (0,1).
	% 		\gamma(h) = (0,-1).
	% 		\gamma(\frac{ 2 }{ 3 }) = (0,1).
	% 		\gamma(1) = (-1,0).
	% 		\function{ a }{ [0,1] }{ [0,3\pi] }.
	% 		a(0) = 0.
	% 		a(\frac{ 1 }{ 3 }) = \frac{ \pi }{ 2 }.
	% 		a(h) = 3 \frac{ \pi }{ 2 }.
	% 		a(\frac{ 2 }{ 3 }) = \frac{ \pi }{ 2 } + 2\pi.
	% 		a(1) = 3\pi
	% 	}
	% 	\is{ a }{ $ an argument determination over $ [0,1] $ $  }
	% 	{
	% 		a $ continuous $
	% 	}
	% }
	
	
	% \proposition{ $ logarithm determinations compositions are holomorphic $ }
	% {
	% 	\letbe
	% 	{
	% 		\Uc \subset \C $ open $.
	% 		\function{ f }{ \Uc }{ \C \nonzero } \in \Hc(\Uc).
	% 		\function{ h }{ \Uc }{ \C } $ logarithm determination of $ log(f)
	% 	}
	% 	\holds
	% 	{
	% 		h \in \Hc(\Uc).
	% 		h'(z) = \frac{ f'(z) }{ f(z) }
	% 	}
	% 	\demonstration
	% 	{
	% 		\all{ z \in \Uc }
	% 		{
	% 			\be{ D }{ \D(f(z),\frac{ 1 }{ 2 }\abs{ f(z) } }.
	% 			D $ no corta muchas semirectas $.
	% 			\ex{ \function{ l }{ Im(f) }{ \C } }
	% 			{
	% 				l $ logarithm determination $
	% 			}.
	% 			\ex{ V \subset \C }
	% 			{
	% 				V $ neighborhood of $ z.
	% 				f(V) \subset D
	% 			}.
	% 			\definedFunction{ \bar{h} }{ D }{ \C }{ z }{ l(f(z)) }.
	% 			f $ hol $, l $ hol $ \imp \bar{h} $ hol in $ z.
	% 			h|_V, \bar{h} $ two determination of logarithm over $ V.
	% 			\ex{ k \in \Z }
	% 			 {
	% 			 	h(z) = \bar{h}(z) + 2k\pi i
	% 			 } 

	% 			h $ hol $ \logicAnd h'(z) = \frac{ f'(z) }{ f(z) }
	% 		}
	% 	}
	% }
	
	
	% Local complex integration\\

	% \definition{ $ Curve $ }
	% {
	% 	\letbe
	% 	{
	% 		A \subset \R.
	% 		\function{ \gamma }{ A }{ \C }
	% 	}
	% 	\then{ \gamma }{ $ a curve $ }
	% 	{
	% 		A $ interval $.
	% 		\gamma $ continuous over $ A
	% 	}
	% 	\denote
	% 	{
	% 		\gamma(A) \as \gamma^*.
	% 		\gamma(a)=\gamma(b) \as \gamma $ closed $
	% 		\gamma \in \C^1(A) \as \gamma $ C^1 $.
	% 		C^1 $a trozos$
	% 	}
	% }


	% \definition{ $ parametrized change $ }
	% {
	% 	\letbe
	% 	{
	% 		\function{ \gamma }{ [a,b] }{ \C } $ curve $.
	% 		\function{ \phi }{ [c,d] }{ [a,b] }
	% 	}
	% 	\then{ \phi }{ $ parametrization change $ }
	% 	{
	% 		\phi $inyective$
	% 	}
	% 	\denote
	% 	{
	% 		\eta \as \gamma \circ \phi
	% 	}
	% }
	
	
	% \definition{ $ Inverse curve $ }
	% {
	% 	\letbe
	% 	{
	% 		\function{ \gamma }{ [a,b] }{ \C } $ curve $
	% 	}
	% 	\name{ $ inverse curve of $ \gamma }
	% 	{
	% 		\definedFunction{ \gamma^{-1} }{ [a,b] }{ \C }{ t }{ a + b - t }
	% 	}
	% }
	
	
	% \definition{ $ length of curve $ }
	% {
	% 	\letbe
	% 	{
	% 		\function{ \gamma }{ [a,b] }{ \C } $ curve $
	% 	}
	% 	\name{ $ length of $ \gamma }
	% 	{
	% 		l(\gamma) = \integral{ \abs{ \gamma'(t) } }{ dt }{ a }[ b ]		
	% 	}
	% }
	
	% Suma directa de curvas hacer \\

	% \definition{ $ Line integral $ }
	% {
	% 	\letbe
	% 	{
	% 		\function{ \gamma }{ [a,b] }{ \C } $ curve $.
	% 		\function{ f }{ \C }{ \C } $ continuous over $ \gamma^*
	% 	}
	% 	\name{ $ line integral of $ \gamma }
	% 	{
	% 		\integral{ f(\gamma(t)) \gamma'(t) }{ dt }{ a }[ b ].
	% 		\integral{ u + vi }{ dx }{ a }[ b ] = \integral{ u }{ dx }{ a }[ b ] + i \integral{ v }{ dx }{ a }[ b ]
	% 	}
	% 	\denote
	% 	{
	% 		\integral{ f }{ dx }{ \gamma }.
	% 		\gamma C^1 $ a trozos $ \integral{ f }{ dx }{ \gamma } = \summation{ \integral{ f }{ dx }{ \gamma_i } }{ i }[ 1 ][ r ]
	% 	}
	% }
	
	
	% \proposition{ $ integral decomposition $ }
	% {
	% 	\letbe
	% 	{
	% 		\function{ f }{ \C }{ \C }.
	% 		u,v $ components of $ f
	% 	}
	% 	\holds
	% 	{
	% 		\integral{ f }{ dx }{ \gamma } = \integral{ u - c }{ dx }{ \gamma } + i \integral{ u + v }{ dx }{ \gamma }
	% 	}
	% 	\demonstration
	% 	{
	% 		$ substitution $
	% 	}
	% }
	
	
	% \example{ $ exponential centered in a $ }
	% {
	% 	\letbe
	% 	{
	% 		\definedFunction{ f }{ \C }{ \C }{ z }{ \frac{ 1 }{ z-a } }.
	% 		\definedFunction{ \gamma }{ [0,2\pi] }{ \C }{ t }{ a + \exp( ikt ) }
	% 	}
	% 	\holds
	% 	{
	% 		\integral{ f }{ dx }{ \gamma } = 2\pi i k
	% 	}
	% }
	
	
	% \proposition{ $ Properties of line integrals $ }
	% {
	% 	\letbe
	% 	{
	% 		\function{ f,g }{ \C }{ \C }.
	% 		\function{ \gamma }{ [a,b] }{ \C } $ curve $.
	% 		\function{ \eta }{ [c,d] }{ \C } $ reparametrization of $ \gamma
	% 	}
	% 	\holds
	% 	{
	% 		\integral{ af + bg }{ dx }{ \gamma } = a \integral{ f }{ dx }{ \gamma } + b \integral{ g }{ dx }{ \gamma }.
	% 		\integral{ f }{ dx }{ \gamma } = \integral{ f }{ dx }{ \eta }.
	% 		\integral{ f }{ dx }{ \gamma^{-1} } = - \integral{ f }{ dx }{ \gamma }.
	% 		\abs{ \integral{ f }{ dx }{ \gamma } } \leq \integral{ \abs{ f(x) } }{ \abs{ dx } }{ \gamma }.
	% 		\abs{ \integral{ f(z) }{ dz }{ \gamma } } \leq \sup_{z \in \gamma} \abs{ f(z) } l(\gamma)
	% 	}
	% 	\demonstration
	% 	{
	% 		$3:$.
	% 		g(t) = u(t) + iv(t).
	% 		\be{ re^{i\theta} }{ \integral{ g(t) }{ dt }{ a }[ b ] }.
	% 		\abs{ \integral{ g(t) }{ dt }{ a }[ b ] } = r = \exp( -i \theta )\integral{ g(t) }{ dt }{ a }[ b ].
	% 		\integral{ \exp( -i\theta )g(t) }{ dt }{ a }[ b ].

	% 		\abs{ \integral{ g(t) }{ dt }{ a }[ b ] } = r = Re r.
	% 		Re r = \integral{ Re \exp( -i\theta )g(t) }{ dt }{ a }[ b ].
	% 		$ real integral $ \imp Re r \leq \integral{ \abs{ \exp( -i\theta )g(t) } }{ dt }{ a }[ b ] = \integral{ \abs{ g(t) } }{ dt }{ a }[ b ].

	% 		$4):$.
	% 		\abs{ \integral{ f(z) }{ dz }{ \gamma } } = \abs{ \integral{ f(\gamma(t))\gamma'(t) }{ dt }{ a }[ b ] } \leq \integral{ \abs{ f(\gamma(t)) }\abs{ \gamma'(t) } }{ dt }{ a }[ b ].

	% 		$5):$.
	% 		\be{ M }{ \sup_{z \in \gamma^* } \abs{ f(z) }.

	% 		\abs{ \integral{ f(z) }{ dz }{ \gamma } } \leq \integral{ \abs{ f(\gamma(t)) }\abs{ \gamma'(t) } }{ dt }{ a }[ b ] \leq M \integral{ \abs{ \gamma'(t) } }{ dt }{ a }[ b ] = M l(\gamma)
	% 	}
	% }
	
	
	















































	% \problem{ $ 9.Root determination $ }
	% {
	% 	\letbe
	% 	{
	% 		\definedFunction{ \sqrt{ - } }{ \C \setminus [0,\infty) }{ \C }{ -1 }{ i }.
	% 		\definedFunction{ f }{ \C \setminus [0,\infty) }{ \C }{ z }{ \sqrt{ 3z + 2 } }
	% 	}
	% 	\showthat
	% 	{
	% 		$express$ \sqrt{ - } $ in terms of $ log $ and $ \arg
	% 	}
	% 	\demonstration
	% 	{
	% 		z^w = \exp( w \log( z ) ) = \exp( w(ln|z|+i\arg(z) ).

	% 		z^{\frac{ 1 }{ 2 }) = \exp( \frac{ 1 }{ 2 }(ln|z|+i\arg(z)) ).

	% 		\sqrt{ -1 } = i \imp \exp( \frac{ 1 }{ 2 }(ln|-1|+iarg(-1) ) = -i \imp arg(-1) = \frac{ \pi }{ 2 }?.

	% 		arg(z) \in (0,2\pi),(4\pi,6\pi)---.

	% 		$la raiz divide el argumento por dos$.

	% 		arg(z) \in (0,2\pi) \imp arg(\sqrt{ z }) \in (0,\pi)..


	% 		w = 3z + 2 \imp z = \frac{ w-2 }{ 3 }.

	% 		\C \setminus [0,\infty) \imp \C [-2,\infty) \imp \C [-\frac{ 2 }{ 3 },\infty).

	% 		f $ defined over $ \C \setminus [-\frac{ 2 }{ 3 },\infty).

	% 		\im( f ) = \set{ z \in \C \with Im z > 0 }.

	% 		f(\frac{ i-2 }{ 3 }) = \sqrt{ i } = \sqrt{ \abs{ i } }\exp( \frac{ 1 }{ 2 }arg(i) ) = \exp( i \frac{ \pi }{ 4 } = \frac{ \sqrt{ 2 } }{ 2 }(1+i) )
	% 	}
	% }
	
	
	% \proposition{ $ Calculus fundamental theorem $ }
	% {
	% 	\letbe
	% 	{
	% 		\function{ \gamma }{ [a,b] }{ \C } $ open curve $.
	% 		\Omega \subset \C $ open $ \suchthat \gamma^* \subset \Omega.
	% 		\function{ f }{ \Omega }{ \C }
	% 	}
	% 	\holds
	% 	{
	% 		\all{ \function{ F }{ \Omega }{ \C } }[ F' = f ]
	% 		{
	% 			\integral{ f(z) }{ dz }{ \gamma } = F(\gamma(b))-F(\gamma(a))
	% 		}
	% 	}
	% 	\demonstration
	% 	{
	% 		$be $F(\gamma(t)) = g(t) = u(t) +iv(t).
	% 		F'(\gamma(t)) = g'(t) = u'(t) + iv'(t).
	% 		\integral{ f(z) }{ dz }{ \gamma } = \integral{ f(\gamma(t) \gamma'(t) }{ dt }{ a }[ b ] = \integral{ F'(\gamma(t))\gamma'(t) }{ dt }{ a }[ b ].

	% 		= \integral{ u'(t) }{ dt }{ a }[ b ] + i \integral{ v'(t) }{ dt }{ a }[ b ].

	% 		$CFT over $ \R.

	% 		= u(b) - u(a) + i(v(b)-v(a)) = F(\gamma(b)) - F(\gamma(a))
	% 	}
	% }
	
	
	% \example{ $ polynomial integrals $ }
	% {
	% 	\letbe
	% 	{
	% 		n \in \Z
	% 		\definedFunction{ f }{ \C }{ \C }{ z }{ z^n }.
	% 		a,b \in \C
	% 	}
	% 	\holds
	% 	{
	% 		\all{ \function{ \gamma }{ [a,b] }{ \C } $ curve $ }
	% 		{
	% 			\definedFunction{ F }{ \C }{ \C }{ z }{ \frac{z^{n+1}}{n+1} }.

	% 			F \in \Hc(\C).

	% 			\integral{ f(z) }{ dz }{ \gamma } = F(b) - F(a)
	% 		}
	% 	}
	% }
	
	
	% \example{ $ logarithm primitive $ }
	% {
	% 	\letbe
	% 	{
	% 		\definedFunction{ f }{ \C \nonzero }{ \C }{ z }{ \frac{ 1 }{ z } }
	% 	}
	% 	\holds
	% 	{
	% 		\ex{ a,b \in \C }
	% 		{
	% 			\ex{ \function{ \gamma_1,\gamma_2 }{ [a,b] }{ \C } $ curve $ }
	% 			{
	% 				\integral{ f(z) }{ dz }{ \gamma_1 } \neq \integral{ f(z) }{ dz }{ \gamma_2 }
	% 			}
	% 		}
	% 	}
	% }
	
	
	\thread{ $ Intergal of power series $ }
	{
		\holds
		{
			\integral{ (z-z_0)^n }{ dz }{ |z-z_0| = r } = \Matrix{ 0, n \neq 1. 2\pi i, n = -1 }
		}
		\demonstration
		{
			$n$ \geq 0:.
			\frac{ (z-z_0)^{n+1} }{ n } \in \int (z-z_0)^n.
			\frac{ (z-z_0)^{n+1} }{ n } \in \Hc(\Cc).
			$CFT over closed curve$:.
			\integral{ (z-z_0)^n }{ dz }{ |z-z_0| = r } = 0.
			$n$ \leq -1:.
			\frac{ (z-z_0)^{n+1} }{ n } \in \int (z-z_0)^n.
			\frac{ (z-z_0)^{n+1} }{ n } \in \Hc(\Cc \setminus z_0).
			$CFT over closed curve$:.
			\integral{ (z-z_0)^n }{ dz }{ |z-z_0| = r } = 0.
			n = -1.
			% log(z-z_0) \in \int (z-z_0)^n $ logarithm determination $.
			% \nexists $ logarithm determination over a $ z_0 $ neighborhood $.
			% \definedFunction{ \gamma }{ (0,2\pi) }{ \C }{ t }{ z_0 + re^{it} }.
			% \integral{ \frac{ 1 }{ z-z_0 } }{ dz }{ \gamma } = \integral{ \frac{ rie^{it} }{ re^{it} } }{ dz }{ 0 }[ 2\pi ] = 2\pi i.
		}
	}


	% \proposition{ $ Triangle Cauchy's theorem $ }
	% {
	% 	\letbe
	% 	{
	% 		\Omega \subset \C \suchthat \gamma* \subset \Omega.
	% 		\function{ f }{ \Omega }{ \C } \in \Hc(\Omega \subset \{p\}), \in \Cc(\Omega).
	% 		T \subset \Omega $ closed triangle $
	% 	}
	% 	\holds
	% 	{
	% 		\integral{ f(z) }{ dz }{ \partial T } = 0
	% 	}
	% 	\demonstration
	% 	{
	% 		$idea$.
	% 		$el triangulo se puede dividir en triangulos tan pequeños como quieras $.
	% 		$cogemos $ 4^n $ triangulos con $ diam(T_n) = \frac{ 1 }{ 2^n } diam T, long(\partial T) = \frac{ 1 }{ 2^n } long(\partial T).
	% 		$en un entorno de $ z_0 $ el cociente incremental menos la derivada es menor que epsilon $.
	% 		\abs{ f(z) - f(z_0) - f'(z_0)(z-z_0) } \leq \epsilon|z-z_0|.
	% 		\be{ g(z) }{ f(z) - f(z_0) - f'(z_0)(z-z_0) } $ polinomio de grado 1 $.
	% 		g $ tiene primitiva holomorfa y $ \integral{ g(z) }{ dz }{ \partial T } = 0.
	% 		\abs{ \integral{ f(z) }{ dz }{ \partial T_n } } = \abs{ \integral{ f(z) - g(z) }{ dz }{ \partial T_n } } \leq max_{z\in \partial T_n} \epsilon |z-z_0| long(T_n) = \epsilon diam T_n long(\partial T_n).

	% 		\integral{ f(z) }{ dz }{ \partial T } = 4^n \abs{ \integral{ f(z) }{ dz }{ \partial T_n } } \leq 4^n \epsilon diam T_n long(T_n).
	% 		= 4^n \epsilon \frac{ 1 }{ 2^n }diam T \frac{ 1 }{ 2^n }long(T) = \epsilon diam T long(T) \convergesto 0.

	% 	}
	% }
	
	
	% \proposition{ $ Convex open Cauchy's theorem $ }
	% {
	% 	\letbe
	% 	{
	% 		\Omega \subset \C $ convex open $.
	% 		\function{ f }{ \Omega }{ \C } \in \Hc(\Omega \setminus \{p\}, f \in \Cc(\Omega).
	% 		\function{ \gamma }{ [a,b] }{ \C } $ closed curve$.
	% 		\gamma \subset \Omega
	% 	}
	% 	\holds
	% 	{
	% 		\ex{ F \in \int f }
	% 		{
	% 			F \in \Hc(\Omega)
	% 		}.
	% 		\integral{ f(z) }{ dz }{ \gamma } = 0
	% 	}
	% 	\demonstration
	% 	{
	% 		\all{ a \in \Omega }
	% 		{
	% 			\all{ z \in \Omega }
	% 			{
	% 				\Omega $ convex $\imp [a,z] \subset \Omega.

	% 				\definedFunction{ F }{ \Omega }{ \C }{ z }{ \integral{ f(u) }{ du }{ [a,z] } }.

	% 				\all{ z_0 \in \Omega }[ z_0 \neq z ]
	% 				{
	% 					F(z) - F(z_0) = \integral{ f(u) }{ du }{ [a,z] } - \integral{ f(u) }{ du }{ [a,z_0] }.

	% 					$Triangle Cauchy theorem $.
	% 					\integral{ f(u) }{ du }{ [z_0,z] }.

	% 					\abs{\frac{ F(z) - F(z_0) }{ z - z_0 } - f(z_0)} = \abs{ \frac{ F(z) - F(z_0) - f(z_0)(z-z_0) }{ z - z_0 } }.

	% 					\integral{ f(z_0) }{ du }{ [z_0,z] } = f(z_0)(z-z_0)
	% 					= \frac{ 1 }{ |z-z_0| }\abs{ \integral{ f(u) - f(z_0) }{ du }{ [z_0,z] } }.

	% 					\leq \frac{ 1 }{ \abs{ z-z_0 } }\sup_{u \in [z_0,z]}\abs{ f(u) - f(z_0) }long([z_0,z]) = \sup_{u \in [z_0,z]} \abs{ f(u) - f(z_0) }.

	% 					f \in \Cc(\Omega) \imp f(u) \convergesto f(z_0).

	% 					\sup_{u \in [z_0,z]} \abs{ f(u) - f(z_0) } \convergesto 0.
	% 				}
	% 			}
	% 		}		
	% 	}
	% }
	
	
	
	
	
	
	
	
	% \definition{ $ Index $ }
	% {
	% 	\letbe
	% 	{
	% 		\function{ \gamma }{ [a,b] }{ \C } $ closed curve $.
	% 		z_0 \nin \gamma^*.
	% 		\phi(t) $ continuous determination of argument $.
	% 	}
	% 	\name{ $ index of $ \gamma $ respect $ z_0 }
	% 	{
	% 		Ind(\gamma,z_0) = \frac{ \phi(b) - \phi(a) }{ 2\pi }
	% 	}
	% }
	
	
	% \proposition{ $ Index integral formula $ }
	% {
	% 	\letbe
	% 	{
	% 		h $ logarithm determination of $ log(\gamma(t) - z_0).
	% 		\gamma \Cc^1 $ curve $
	% 	}
	% 	\holds
	% 	{
	% 		Ind(\gamma,z_0) = \frac{ h(b)-h(a) }{ 2\pi i }	.

			
	% 		\gamma $ closed $, z_0 \nin \gamma^* \imp Ind(\gamma,z_0) = \frac{ 1 }{ 2\pi i }\integral{ \frac{ 1 }{ z-z_0 } }{ dz }{ \gamma }.
	% 	}
	% 	\demonstration
	% 	{
	% 		h'(t) = \frac{ (\gamma(t) - z_0)' }{ \gamma(t) - z_0 } = \frac{ \gamma'(t) }{ \gamma(t) - (z_0) }.

	% 		Ind(\gamma,z_0) = \frac{ h(b) - h(z) }{ 2 \pi i }.

	% 		$CFT$:.

	% 		\frac{ 1 }{ 2\pi i } \integral{ h'(t) }{ dt }{ a }[ b ] = \frac{ 1 }{ 2\pi i } \integral{ \frac{ \gamma'(t) }{ \gamma(t) - z_0 } }{ dt }{ a }[ b ] = \frac{ 1 }{ 2\pi i }\integral{ \frac{ 1 }{ z-z_0 } }{ dz }{ \gamma }.

	% 	}
	% }


	% \proposition{ $ Index properties $ }
	% {
	% 	\letbe
	% 	{
	% 		\function{ \gamma }{ [a,b] }{ \C } $ closed curve $.
	% 		z_0 \nin \gamma^*
	% 	}
	% 	\holds
	% 	{
	% 		Ind(\gamma^{-1},z_0) = - Ind(\gamma,z_0).
	% 		Ind(\gamma,z)  $ constant over all connected of $ \C \setminus \gamma^*.
	% 		Ind(\gamma,z) = 0 $ over non-bounded connected component of $ \C \setminus \gamma^*
	% 	}
	% 	\demonstration
	% 	{
	% 		$ exercise $
	% 	}
	% }
	
	
	\thread{ $ Integral formula of Cauchy over convex open sets $ }
	{
		\letbe
		{
			\gamma $ closed curve $.
			\Omega \subset \C $ convex open $ \suchthat \gamma^* \subset \Omega.
			f \in \Hc(\Omega).
			z \nin \gamma^*.
		}
		\holds
		{
			\frac{ 1 }{ 2\pi i }\integral{ \frac{ f(\omega) }{ \omega - z } }{ d\omega }{ \gamma } = f(z) Ind(\gamma,z)
		}
		\demonstration
		{
			\all{ z \nin \gamma^* }
			{
				\definedFunction{ \tilde{f} }{ \Omega }{ \C }{ \omega }{ \begin{cases} \frac{ f(\omega) - f(z) }{ \omega - z } & \omega \neq z \\ f'(z) & \omega = z \end{cases} }.

				\tilde{f} \in \Cc(\Omega).

				\tilde{f} \in \Hc(\Omega \setminus \{z\}).

				$Cauchy's theorem:$.

				\integral{ \tilde{f}(w) }{ dw }{ \gamma } = 0.

				z \nin \gamma^* \imp \omega \neq z.

				\integral{ \tilde{f}(\omega) }{ d \omega }{ \gamma } = \integral{ \frac{ f(\omega) - f(z) }{ \omega - z } }{ d \omega }{ \gamma }.

				= \integral{ \frac{ f(\omega) }{ \omega - z } }{ d \omega }{ \gamma } - f(z) \integral{ \frac{ 1 }{ \omega - z } }{ d \omega }{ \gamma } =.

				= \integral{ \frac{ f(\omega) }{ \omega - z } }{ d \omega }{ \gamma } - f(z)2\pi i Ind(\gamma,z) = 0.

				\frac{ 1 }{ 2\pi i }\integral{ \frac{ f(\omega) }{ \omega - z } }{ d\omega }{ \gamma } = f(z) Ind(\gamma,z)

			}
		}
	}


	\thread{ $ Mean property $ }
	{
		\letbe
		{
			\Omega \subset \C $ open $.
			f \in \Hc(\Omega).
			D(a,r) \subset \Omega.
		}
		\holds
		{
			f(a) = \frac{ 1 }{ 2\pi }\integral{ f(a+re^{i\theta}) }{ d \theta }{ 0 }[ 2\pi ].
		}
		\demonstration
		{
			$Integral formula$:.

			f(a) = \frac{ 1 }{ 2\pi i }\integral{ \frac{ f(z) }{ z-a } }{ dz }{ \partial D(a,r) }.

			\be{ \gamma }{ \partial D(a,r) }.

			\gamma(\theta) = a + re^{i \theta}.

			\gamma'(\theta) = rie^{i \theta}.

			f(a) = \frac{ 1 }{ 2\pi i }\integral{ \frac{ f(a+re^{i\theta} }{ re^{i \theta} }re^{i \theta} }{ d \theta }{ 0 }[ 2\pi ].
		}
	}
	
	
	\thread{ $ Independence of $ \gamma }
	{
		\letbe
		{
			\Omega \subset \C $ open $.
			f \in \Hc(\Omega).
			\gamma, \tilde{\gamma} $ closed curve $ \suchthat \gamma^* \subset \Omega.
			z \in \Omega \suchthat Ind(\gamma,z) = Ind(\tilde{\gamma},z).
		}
		\holds
		{
			\integral{ \frac{ f(\omega) }{ \omega-z } }{ d \omega }{ \gamma } = \integral{ \frac{ f(\omega) }{ \omega-z } }{ d \omega }{ \tilde{\gamma} }
		}
		\demonstration
		{
			$ no demonstration $
		}
	}
	
	
	
	
	
	\thread{ $ Integral formula application $ }
	{
		\letbe
		{
			\gamma $ pasa por en medio de i,-i y rodea 1/2 $.
		}
		\holds
		{
			\integral{ \frac{ \cos( \frac{ pi }{ 2 }\omega ) }{ \omega } }{ d \omega }{ \gamma }.
			\integral{ \frac{ \cos( \frac{ pi }{ 2 }\omega ) }{ \omega - \frac{ 1 }{ 2 }} }{ d \omega }{ \gamma }.
		}
		\demonstration
		{
			f(\omega) = \cos( \frac{ \pi }{ 2 }\omega ) \in \Hc(\C) $ convex $.

			\integral{ \frac{ \cos( \frac{ pi }{ 2 }\omega ) }{ \omega } }{ d \omega }{ \gamma } = 2 \pi i 1 1 = 2\pi i.

			\integral{ \frac{ \cos( \frac{ pi }{ 2 }\omega ) }{ \omega - \frac{ 1 }{ 2 } } }{ d \omega }{ \gamma } = 2\pi i \frac{ \sqrt{ 2 } }{ 2 } (-1).
			
			\integral{ \frac{ \cos( \frac{ pi }{ 2 }\omega ) }{ \omega - i } }{ d \omega }{ \gamma } = 2\pi i f(i) 0 = 0.

			\integral{ \frac{ \cos( \frac{ pi }{ 2 }\omega ) }{ \omega(\omega^2 + 4) } }{ d \omega }{ \gamma } = \integral{ \frac{ \cos( \frac{ pi }{ 2 }\omega ) }{ \omega(\omega+2i)(\omega-2i) } }{ d \omega }{ \gamma }.

			2i , -2i \nin \gamma^*.

			\integral{ \frac{ \cos( \frac{ \pi }{ 2 }\omega )/(\omega^2 + 4) }{ \omega } }{ d \omega }{ \gamma } = 2 \pi i \frac{ 1 }{ 4 } 1. 
		}
	}


	\thread{ $ Basic exercises $ }
	{
		\letbe
		{
			\definedFunction{ \gamma }{ [0,\pi] }{ \C }{ t }{ 1 - cos(t) + isin(t) }.
			\definedFunction{ f }{ \C }{ \C }{ z }{ \frac{ 1 }{ 2+z^2 } }.
			\definedFunction{ f_2 }{ \C }{ \C }{ z }{ \frac{ sin(z) }{ z^2 } }.
			\definedFunction{ f_3 }{ \C }{ \C }{ z }{ \frac{ (1+z)^n }{ z^{m+1} } }
		}
		\holds
		{
			\abs{\integral{ f(z) }{ dz }{ \gamma }} \leq \pi.
			\abs{\integral{ f_2(z) }{ dz }{ \gamma }} \leq 2\pi(sin(1)^2+sinh(1)^2).
		}
		\demonstration
		{
			\abs{\integral{ f(z) }{ dz }{ \gamma }} \leq \sup_{z \in \gamma^*} \abs{ f(z) } long(\gamma).

			\min_{z\in \gamma^*} |2+z^2| = 1 \imp \sup_{z \in \gamma^*} = 1.

			\sup_{z \in \gamma^*} \abs{ f(z) } long(\gamma) \leq \pi.

			\abs{ sin(z) }^2 = \sin( x )^2 + \sin( hy ) \leq (\sin(1))^2 + (sinh(1))^2.

			\abs{\integral{ f_2(z) }{ dz }{ \gamma }} \leq 2\pi(sin(1)^2+sinh(1)^2).

			\abs{\integral{ f_3(z) }{ dz }{ \gamma }} $Formula binomio$.
		}
	}
	
	
	Power Series II\\

	\thread{ $ Analytic $ }
	{
		\letbe
		{
			\function{ f }{ \C }{ \C }
		}
		\then{ f }{ $ analytic $ }
		{
			$exists power series development of $ f.
		}
	}
	
	
	

	\thread{ $ Holomorphic functions are power series $ }
	{
		\letbe
		{
			\Omega \subset \C $ open $.
			f \in \Hc(\C).
			D(a,R) \subset \Omega.
		}
		\holds
		{
			f(z) = \summation{ c_n(z-a)^n }{ n }[ 0 ] $ where $.

			c_n = \frac{ 1 }{ 2\pi i }\integral{ \frac{ f(\omega) }{ (\omega-a)^{n+1} } }{ d \omega }{ \partial D(a,\rho) }, \rho < R.

			f \in \Cc^{\infty}(\Omega).	
		}
		\demonstration
		{
			\all{ \rho \in (0,R) }
			{
				\definedFunction{ \gamma }{ (0,2\pi) }{ \C }{ t }{ a + \rhoe^{it} }.

				\all{ z \in \Omega }[ \abs{ z - a } < \rho ]
				{
					Ind(\gamma,z) = 1.

					$Integral formula$:.

					f(z) = \frac{ 1 }{ 2\pi i }\integral{ \frac{ f(\omega) }{ \omega-z) } }{ d \omega }{ \gamma } = *.
				}.

				\frac{ 1 }{ \omega-z } = \frac{ 1 }{ \omega-a }\frac{ 1 }{ 1 - \frac{ z-a }{ \omega-a } }.

				\abs{ \frac{ z-a }{ \omega-a } } < 1.

				\frac{ 1 }{ 1 - \frac{ z-a }{ \omega-a } } = \summation{ (\frac{ z-a }{ \omega -a })^n }{ n }[ 0 ].

				\frac{ 1 }{ \omega-z } = \summation{ (\frac{ z-a }{ \omega -a })^{n+1} }{ n }[ 0 ].

				$UCI theorem:$.

				f(z) = * = \frac{ 1 }{ 2\pi i }\integral{ f(\omega)\summation{ \frac{ (z-a)^n }{ (\omega-a)^{n+1} } }{ n }[ 0 ] }{ d \omega }{ \gamma }.

				= \frac{ 1 }{ 2\pi i }\summation{ \integral{ \frac{ f(\omega) }{ (\omega-a)^{n+1} } }{ d \omega }{ \abs{ \omega-a }=\rho } (z-a)^n }{ n }[ 0 ].

				= \summation{ c_{n,\rho}(z-a)^n }{ n }[ 0 ].

				\be{ c_{n,\rho} }{ \frac{ 1 }{ 2\pi }\integral{ \frac{ f(\omega) }{ (\omega-a)^{n+1} } }{ d \omega }{ \abs{ \omega - a } = \rho } }.

				$ Taylor coefficients of power series:$.

				c_{n,\rho} = \frac{ f^{n)}(a) }{ n! }.

				$ no dependency of rho $.


			}

			c_n = \frac{ 1 }{ 2\pi i }\integral{ \frac{ f(\omega) }{ (\omega-a)^{n+1} } }{ d \omega }{ \partial D(a,\rho) }, \rho < R.

		}
	}


	\thread{ $ Integral formula for derivatives $ }
	{
		\letbe
		{
			\Omega \subset \C $ open $.
			D(a,r) \subset \Omega.
			f \in \Hc(\Omega).
		}
		\holds
		{
			f^{n)}(a) = \frac{ n! }{ 2\pi i }\integral{ \frac{ f(\omega) }{ (\omega-a)^{n+1} } }{ d \omega }{ \abs{ \omega-a } = \rho }.
		}
		\demonstration
		{
			$ Taylor coefficients substitution $.
		}
	}
	
	
	\thread{ $ Integral formula for derivatives application $ }
	{
		\letbe
		{
			statements.
		}
		\holds
		{
			\integral{ \frac{ ze^z }{ (z-1)^2 } }{ dz }{ \abs{ z-1 } = 1 } = \frac{ 2\pi i }{ 1! }f'(1) = 2\pi i 2e
		}
		\demonstration
		{
			demonstration.
		}
	}
	
	
	
	
	
	
	
	
	
	
	% \example{ $ 3. Polynomial zeros integration $ }
	% {
	% 	\letbe
	% 	{
	% 		p \in \polynomial( \C ).
	% 		n = \cardinal Z(p).
	% 		R \in \R \suchthat Z(p) \subset \D(0,R)
	% 	}
	% 	\holds
	% 	{
	% 		\integral{ \frac{ p'(z) }{ p(z) } }{ dz }{ |z|=R } = 2\pi i n.
	% 	}
	% 	\demonstration
	% 	{
	% 		p(z) = c \product{ (z-z_i) }{ i }[ 1 ][ n ] \suchthat z_i \in Z(p).
	% 		p'(z) = \summation{ c \product{ (z-z_i) }{ i }[ 1, i \neq j ][ n ] }{ j }[ 1 ][ n ].
	% 		= c \product{ (z-z_i)(\summation{ \frac{ 1 }{ (z-z_i) } }{ i }[ 1 ][ n ] }{ i }[ 1 ][ n ].
	% 		\frac{ p'(z) }{ p(z) } = \frac{c \product{ (z-z_i)\summation{ \frac{ 1 }{ z-z_i } }{ i }[ 1 ][ n ] }{ i }[ 1 ][ n ]}{c \product{ (z-z_i) }{ i }[ 1 ][ n ].
	% 		\integral{ \summation{ \frac{ 1 }{ z-z_i } }{ i }[ 1 ][ n ] }{ dz }{ |z|=R } = \summation{ \integral{ \frac{ 1 }{ z-z_j } }{ dz }{ |z|=R } }{ i }[ 1 ][ n ].
	% 		= \summation{ 2\pi i }{ j }[ 1 ][ n ] = 2 \pi i n.
	% 	}
	% }
	











	
	
	
	
	
	
	




















	
	

}


\end{document}