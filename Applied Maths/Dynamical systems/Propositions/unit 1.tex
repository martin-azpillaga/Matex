
\documentclass[../Main/main]{subfiles}


\begin{document}


\unit{ $ One-dimensional discrete dynamical systems $ }
{
	\introduction
	{ 
		$introduction\\$ 
	}

	\proposition{ $ Fixed points theorem $ }
	{
		\letbe
		{
			I \subset \R $ open $.
			\function{ f }{ I }{ I } $ differentiable $.
			x \in I
		}
		\holds
		{
			|f'(x)| < 1 \imp x $ attractive $.
			|f'(x)| > 1 \imp x $ repulsive $
		}
		\demonstration
		{
			demonstration.
		}
	}
	
	
	\proposition{ $ Attractiveness of periodic points does not involve the chosen point $ }
	{
		\letbe
		{
			( M, \N, f ) $ functional dynamical system $.
			x \in M $ n-periodic point $.
			\family{ x_i }{ i }[ 1 ][ r ] $ orbit of $ x
		}
		\holds
		{
			x $ attractive $ \ifandonlyif \all{ x' \in o(x) }
			{
				x' $ attractive $
			}
		}
		\demonstration
		{
			\all{ x' \in o(x) }
			{
				f^{n'}(x') = \product{ f'(x_i) }{ i }[ 1 ][ r ] = f^{n'}(x)
			}
		}
	}


	\proposition{ $ Partition of attraction set $ }
	{
		\letbe
		{
			( M, \N, f ) $ functional dynamical system $.
			x $ n-periodic point $.
			o(x) $ orbit of $ x
		}
		\holds
		{
			\all{ x' \in o(x) }
			{
				\ex{ \Uc \subset M $ open $ }{ \all{ y \in \Uc }
				{
					f^n(y) \convergesto x'
				} }
			}
		}
		\demonstration
		{
			demonstration.
		}
	}
	
	
	\proposition{ $ Homeomorphisms are monotonous $ }
	{
		\letbe
		{
			\function{ f }{ \R }{ \R } $ homeomorphism $
		}
		\holds
		{
			f $ monotonous $
		}
		\demonstration
		{
			$ no demonstration $
		}
	}


	\proposition{ $ Homeomorphisms and n-periodic points $ }
	{
		\letbe
		{
			\function{ f }{ \R }{ \R } $ homeomorphism $
			( M, T, \phi ) $ dynamical system defined by $ f
		}
		\holds
		{
			\all{ n \in \N }
			{
				\nexists \; x \in M \suchthat x $ n-periodic point $
			}
		}
		\demonstration
		{
			$graphically $
		}
	}


	\proposition{ $ Sarkovskii's theorem $ }
	{
		\letbe
		{
			\function{ f }{ I }{ I }.
			( M, T, \phi ) $ dynamical system $
		}
		\holds
		{
			\ex{ x \in M }
			{
				o(x) $ k-period $
			}.

			\imp \all{ l \in \N }[ l > k ]
			{
				\ex{ x' \in M }
				{
					x' $ l-period $
				}
			}
		}
	}


	\proposition{ $ Characterization of sella-node bifurcation points $ }
	{
		\letbe
		{
			I \subset \R
			\family*{ \function{ f_\lambda }{ I }{ I } }{ \lambda \in \Lambda }.
			x \in I
		}
		\holds
		{
			x $ SN bifurcation point $ \ifandonlyif f(x) = x, f'(x) = 1.

			\partialderivative{ f }{ \lambda } \neq 0.

			f''(x) \neq 0
		}
		\demonstration
		{
			demonstration.
		}
	}
	
	\definition{ $ Pitchfork bifurcation $ }
	{
		\letbe
		{
			( M, T, f_\lambda ) $ dynamical system family $.
			x \in M
		}
		\then{ x }{ $ pitchfork bifurcation $ }
		{
			x $ fixed point $.
			$ born of 2 fixed points $.
		}
	}


	
	\proposition{ $ Characterization of Pitchfork bifurcations $ }
	{
		\letbe
		{
			( M, T, f_\lambda ) $ dynamical system family $.
			x \in M
		}
		\holds
		{
			x $ Pitchfork $ \ifandonlyif.

			f(x) = x.

			f'(x) = 1.

			\partialderivative{ f }{ x^2 } = 0.

			\partialderivative{ f }{ \lambda } = 0
		}
		\demonstration
		{
			$ no demonstration $
		}
	}
	

}


\end{document}