


\documentclass[12pt]{article}
\usepackage{amsmath,amssymb,amsfonts,amscd,amsthm}
\usepackage[latin1]{inputenc}
\usepackage{color,soul}
\setulcolor{red}

\setlength{\textwidth}{16cm} \setlength{\textheight}{22cm}
\setlength{\hoffset}{-14mm} \setlength{\voffset}{-10mm}

\newcommand{\C}{\mathbb C}
\newcommand{\R}{\mathbb R}
\newcommand{\Q}{\mathbb Q}
\newcommand{\Z}{\mathbb Z}
\newcommand{\N}{\mathbb N}

\newcommand{\M}{\mathcal M}


\begin{document}                    %  aqui comen�a el document

%---------------------------
{\bf Problema 1.8.11}     %   {\bf  text   }  escriu  text en negreta.

\vspace{2mm}      %  Deixem un espai entre par�grafs.

Es diu {\em tra\c{c}a d'una matriu} quadrada $A=(a_i^j)$ a la suma       %  {\em    }  entorn amb �mfasi en el text
dels elements de la seva diagonal $tr (A):=\sum_ i a_i^i$.
Demostreu les propietats seg{\"u}ents:

\begin{enumerate}                                                       %  entorn d'una lista, cada  item va numerat correlativament.
\item $tr(A+B)= tr(A)+tr(B)$, on $A$ i $B$ s�n del mateix ordre.

\item  $tr(\lambda A)=\lambda tr(A)$, on $\lambda$ �s un escalar.

\item $ tr(AB)=tr(BA)$, on $A$ i $B$ s�n del mateix ordre.

\item Si $A\in M_{m\times n}(\mathbb{R})$, $B \in M_{n\times k}
(\mathbb{R})$ i $ C \in M_{k\times m} (\mathbb{R})$, aleshores
$$
tr(ABC)=tr(BCA)=tr(CAB).
$$
En el cas $m=n=k$ es compleix que $tr(ABC)=tr(BAC)$?

\item Proveu que no existeixen matrius quadrades $A$, $B$,  tals
que $AB-BA=I$.
\end{enumerate}                                                     %  aqui acaba l'entorn  enumerate.

\vspace{5mm}

{\bf Resoluci� de 1.} Siguin $A=(a_i^j)$ , $B=(b_i^j)$. Aleshores,
$$
tr (A) + tr (B) = \sum_ i a_i^i + \sum_ i b_i^i = \sum_ i (a_i^i +
b_i^i) = tr (A+B)
$$

\vspace{5mm}

{\bf Resoluci� de 2.} Siguin $A=(a_i^j)$ , $\lambda$ un escalar.
Aleshores,
$$
tr ( \lambda A) = \sum_i \lambda(a_i^i) = \lambda \sum_i a_i^i= \lambda tr (A)           %   COMPLETA
$$

\vspace{5mm}

{\bf Resoluci� de 3.} Siguin $A=(a_i^j)$ , $B=(b_i^j)$. Siguin $AB
= (c_i^j)$, $BA = (d_i^j)$, on  $c_i^j = \sum_u a_u^j b_i^u$, i
$d_i^j = \sum_u b_u^j a_i^u$. Aleshores,
$$
tr (AB) = \sum_i c_i^i = \sum_i \left(\sum_u a_u^ib_i^u\right) = \sum_{i,u} a_u^ib_i^u = \sum_{j,k} a_k^jb_j^k
$$
$$
tr (BA) = \sum_i d_i^i = \sum_i \left(\sum_u b_u^ia_i^u\right) = \sum_{i,u} b_u^ia_i^u = \sum_{k,j} a_k^jb_j^k
$$
i, per tant, $tr (AB) = tr(BA)$.

\vspace{5mm}

{\bf Resoluci� de 4.} Sigui $A=(a_i^j)$ , $B=(b_i^j)$,
$C=(c_i^j)$. Posem $AB = (p_i^j)$, on $p_i^j = \sum_u a_u^j
b_i^u$. Si $ABC = (q_i^j)$, tenim
$$
q_i^j = \sum_u q_u^j c_i^u = \sum_u \left( \sum_v a_v^j
b_u^v\right) c_i^u  = \sum_{u, v} a_v^j b_u^v c_i^u
$$

Per calcular els coeficients de $BCA$ i de $CAB$ nom�s ens cal
canviar el nom dels coeficients, i obtenim respectivament:
$$
\sum_{u, v} b_v^j c_u^v a_i^u, \qquad \sum_{u,v}c_u^ja_v^ub_i^v                 %ESCRIU ELS COEFICIENTS DE CAB
$$

Podem calcular ara la tra\c{c}a d'aquestes matrius:
$$
tr (ABC) = \sum_{i, u, v} a_v^i b_u^v c_i^u, \qquad tr(BCA) =
\sum_{i, u, v} b_v^i c_u^v a_i^u, \qquad tr(CAB) = \sum_{i,u,v}c_u^ia_v^ub_i^v               %ESCRIU ELS SUMATORIS CORRESPONENTS
$$
i veiem que s�n iguals.

\vspace{2mm}

El producte $BAC$ nom�s t� sentit si $m=n=k$. En aquest cas,
$$tr(BAC) = \sum_{i,u,v} b_u^ia_v^uc_i^v $$ que pot ser diferent de $tr(ABC)$.

Un exemple:
$$
A= \begin{pmatrix} 1 & 0 \\ 0 & 0 \end{pmatrix}, \qquad B=
\begin{pmatrix} 0 & 0 \\ 1 & 0 \end{pmatrix},  \qquad C= \begin{pmatrix} 0 & 1 \\ 0 & 0 \end{pmatrix}
$$

\vspace{5mm}

{\bf Resoluci� de 5.} 


Usaremos dos propiedades especificadas anteriormente: $$tr(AB) = tr(BA)$$ $$tr(A+B)=tr(A) + tr(B)$$
Entonces, siendo $A$ y $B$ matrices de orden $n$:
$$
tr(AB-BA) = tr(AB)-tr(BA) = 0
$$
$$
tr(I) = \sum_{i=1}^n a_i^i = \sum_{i=1}^n 1 = n
$$
$$
n \geq 1 \rightarrow tr(I) \geq 1
$$
$$
tr(AB-BA) \neq tr(I)
$$                  % ESCRIU LA RESPOSTA


%---------------------------

\end{document}                                                      % aqu� acaba el document.
