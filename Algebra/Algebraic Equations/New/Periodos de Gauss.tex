
\documentclass[a5paper]{article}
\renewcommand{\baselinestretch}{1.5}
\usepackage{matex}
\usepackage{titlesec}
\renewcommand{\C}{\mathcal{C}}
\titleformat{\section}{\bfseries\filcenter}{}{0pt}{\bfseries\filcenter}




\begin{document}
\titlePage{ $ Estructura algebraica de los periodos de Gauss $ }{ $ Martin Azpillaga $ }
\renewcommand{\blockname}{ Estructura algebraica de los periodos de Gauss }

\abstract
{
	Lo que prosigue es un análisis sobre la estructura algebraica de los números complejos conocidos como periodos de Gauss. El contenido parte de un resumen de los resultados hallados en:\\
	Travesa, Artur: {\it El teorema de Kronecker-Weber}. \\ CSIC Madrid, 2008. \href{http://atlas.mat.ub.es/personals/travesa/Kr-W.pdf}{link}\\
	y continúa con la abstracción de los periodos de Gauss presente en:

	N.J. Wildberger: {\it On the algebraic structure of Gaussian periods}. University of South Wales, 2003. \href{http://web.maths.unsw.edu.au/~norman/papers/GaussianPeriods.pdf}{link}\\

	El objetivo es mostrar la importancia de los periodos de Gauss en la teoria algebraica de números y dar una abstracción para su análisis desde el punto de vista del álgebra abstracta.
}
\newpage
\begin{center} {\bf Motivación} \end{center}
{
Sea $H$ un subgrupo de $(\Z/p\Z)^*$ de índice $n$. Como $(\Z/p\Z)^*$ es abeliano, $H$ es normal y el cociente $(\Z/p\Z)^*/H$ está bien definido. El subgrupo $H$ particiona el conjunto $(\Z/p\Z)^*$ en $n$ clases laterales, las denotaremos por $C_1,...C_n$.\\
Definamos ahora los objetos de nuestro estudio:\\
{\bf Definición.} Periodos de Gauss.

Sea:
\begin{itemize}
\item $H$ subgrupo de $\Z/p\Z^*$ de índice $n$
\item $\{C_1,...,C_n\}$ las clases laterales de $\Z/p\Z^*/H$
\item $i \in \{1,...,n\}$

\end{itemize}
Llamamos $i$-ésimo $n$-periodo de Gauss al numero complejo:
\begin{itemize}
\item $\sumatoryoverset{\zeta^x}{x}{C_i}$
\end{itemize}
donde $\zeta$ es la raiz n-ésima primitiva: $e^{\frac{i\pi}{n}}$.\\
Cuando el valor de $n$ se sobreentienda, denotaremos por $\eta_i$ al $i$-ésimo $n$-periodo de Gauss.\\
El interes por los periodos de Gauss viene razonado por el siguiente teorema:\\
{\bf Teorema.} Determinación de los subcuerpos ciclotómicos.

Sea:
\begin{itemize}
\item $p$ primo impar
\item $\zeta$ raíz p-ésima de la unidad
\item $K$ subcuerpo de $\Q(\zeta)$ tal que $[K:\Q] = n$
\end{itemize}
Entonces, se cumple:

$ K = \Q(\eta_i) \hspace{1cm}	\forall i \in \{1,...,n\}$.\\
En otras palabras, los $n$-periodos de Gauss son elementos primitivos de la única subextensión de $K|\Q$ de grado $n$.

}
\newpage
{
Comenzaremos recordando tres resultados previos:

{\bf Resultado I.} 

Sea:
\begin{itemize}
\item $p$ primo impar
\item $n$ divisor de $p-1$
\end{itemize}

Entonces, se cumple:
\begin{itemize}
\item $\exists !\; H < (\Z/p\Z)^* $ tq $ |(\Z/p\Z)^*:H| = n$.
\end{itemize}

{\bf Resultado II.} 

Sea:
\begin{itemize}
\item $L|k$ extensión de Galois
\item $\theta \in L$ elemento primitivo de $L|k$
\item $K \subset L$ subcuerpo que contiene $k$
\item $a_0,...,a_n$ coeficientes de $Irr(\theta,K)$
\end{itemize}

Entonces, se cumple:
\begin{itemize}
\item $K = k(a_0,...,a_n)$
\end{itemize}

{\bf Resultado III.} 

Sean:
\begin{itemize}
\item $n \in N$
\item $s_k(X_1,...,X_n)$ el k-ésimo polinomio simétrico
\item $t_k(X_1,...,X_n)$ el k-ésimo polinomio de Newton
\item $x_1,...,x_n$ elementos algebraicos sobre $\Q$
\end{itemize}

Entonces, se cumple:
\begin{itemize}
\item $\Q(s_1(x_1,...,x_n),...,s_n(x_1,...,x_n)) = \Q(t_1(x_1,...,x_n),...,t_n(x_1,...,x_n))$
\end{itemize}
}
\newpage
{
	Empecemos ahora con la demostración del teorema:\\
	Sabemos que Gal$(Q(\zeta)|Q)$ es isomorfo a $(\Z/p\Z)^*$ que tiene orden $p-1$. Aplicando el resultado I, sabemos que existe un único subgrupo $H$  de Gal$(Q(\zeta)|Q)$ con índice $n$, y por la correspondencia de Artin, $K$ ha de ser $\Q^H$.\\

	Por el resultado II, el cuerpo $K$ está generado sobre $\Q$ por los coeficientes
	del polinomio $Irr(\zeta,K)(X)$.\\
	Como $Gal (Q(\zeta)|Q)$ es cíclico, $H$ también. Sea $\sigma$ un generador de $H$. Podemos expresar:
	$$ Irr(\zeta,K)(X) = \productoryoverset{(X-\tau(\zeta)}{\tau}{H} = \productory{ (X-\sigma^{jn}(\zeta) }{ j }[ 1 ][ d ] $$
	donde $d$ es $p-1/n$. \\

	Para cada $j \in \{1,...,d\}$, denotemos por $a_j$ a $\sigma^{jn}(\zeta)$. Los coeficientes del polinomio irreductible vienen dados por los polinomios simétricos elementales evaluados sobre los $a_j$, esto es, son los valores: $s_1(a_1,...,a_d),...,s_d(a_1,...,a_d)$.\\
	Por el resultado III, los valores $t_1(a_1,...,a_d),...,t_d(a_1,...,a_d)$ generan el mismo cuerpo sobre $\Q$ y resulta que cada uno de estos es un $n$-periodos de Gauss. Visualmente:
	$$ K = \Q(a_j)_j = \Q(s_j(a_1,...,a_d))_j = \Q(t_j(a_1,...a_d))_j \subset \Q(\eta_j)_j$$

	Ahora, por la definición de periodos de Gauss se sigue que $\sigma(\eta_i) = \eta_i$ de manera que $\eta_i \in K$ para todo $i$ y obtenemos la igualdad: $K = Q(\eta_1, . . . , \eta_n)$\\
	Finalmente, para todo $i,j$ enteros, se satisface la igualdad $\sigma^j(\eta_i) = \eta_{i+j}$. Concluimos que todos los periodos $\eta_i$ son conjugados entre sí y efectivamente: $$K = Q(\eta_i)$$}
{
Nuestro interes en estas paginas es estudiar la estructura de los periodos de Gauss. Para ello, analizaremos las propiedades que los caracterizan para despues abstraerlas y presentar una forma de estudiar los periodos de Gauss mediante álgebra abstracta. Volvamos a empezar:

Sea $H$ el subgrupo de $\Z/p\Z^*$ de índice $n$. Como $(\Z/p\Z)^*$ es abeliano, H es normal y el cociente $(\Z/p\Z)^*/H$ está bien definido. El subgrupo $H$ particiona el conjunto $(\Z/p\Z)^*$ en $n$ clases laterales, las denotaremos por $C_1,...C_n$.

Ahora, podemos considerar $C_1 = H$. Dado un generador $g$ de $\Z/p\Z^*$, para todo conjunto $C \in \{C_1,...,C_n\}$ existe un $k$ natural menor que $n$ de manera que $C = g^kC_1$, y por ello podemos ordenar los conjuntos $C_1,...,C_n$ mediante la regla:
$$C_i = g^iC_1 < C_j = g^jC_j \leftrightarrow i < j$$

Si añadimos el conjunto $C_0 = \{0\}$, definiendo $C_0 < C_i$ para todo $i$, obtenemos una particion totalmente ordenada de $\Z/p\Z$: $\{C_0,...,C_n\}$. Denotaremos esta partición por $\C$.\\

Escojamos tres conjuntos $C_i,C_j,C_k \in \C$ y fijemos un $x_k$ en $C_k$. Podemos considerar la cantidad de veces que $x_k$ aparece como resultado de $x_i+x_j$ donde $x_i$ y $x_j$ varian en $ C_i $ y $ C_j $ respectivamente. El resultado no depende del $x_k$ elegido y por ello tiene sentido denotar por $N_{ij}^k$ a dicha cantidad.

Para todo $i,j$, se cumple la ecuación:

$$ C_iC_j = \sum_{k=0}^n N_{ij}^kC_k $$

que llamaremos la ecuación característica de $\C$.

\newpage

Consideremos ahora, $\Z\C$ el span de $\C$ por $\Z$, es decir:

$$ \Z\C = \{\sum_{i=0}^n z_iC_i | z_i \in \Z, C_i \in \C \} $$

Gracias a la ecuación característica, $\Z\C$ con la suma y el producto ordinarios definidos sobre los coeficientes $z_i$, tiene estructura de anillo abeliano con unidad $C_0$.\\
Podemos considerar para cada $C_i$ su inverso respecto la suma en $\Z\C$. Denotemoslo por $C_i^*$. Todos los $C_i^*$ pertenecen a $\C$ y podemos considerar la aplicación:

$$\definedfunction{ * }{ \C }{ \C }{ C_i }{ C_i^* }$$

que resulta ser un automorfismo sobre $\C$.

Hemos sido capaces de definir un conjunto de numeros naturales $\{N_{ij}^k\}$ que definen la ecuación carácterística y un automorfismo $*$ sobre $\C$. Estos dos conceptos quedan relacionados mediante la siguiente propiedad:
$$N_{ij}^0 > 0 \leftrightarrow C_j = C_i^*$$

Ahora, podemos abstraer estas propiedades y construir una nueva estructura algebraica del cual los periodos de Gauss formaran un ejemplo importante. 
}
\newpage
{
\begin{center}{\bf Ensamblajes}\end{center}

Abstraeremos las propiedades vistas bajo la definición de ensamblaje, {\it assembly} según el autor original:

{\bf Ensamblaje}

Sea:
\begin{itemize}

\item $\C = \{C_i\}_{i=0}^n$ un conjunto totalmente ordenado.
\item $N = \{N_{ij}^k\}_{i,j,k=0}^n \subset \N$.

\item $* :  \C \longrightarrow \C $.

\end{itemize}

Entonces, $(\C,N,*)$ es un ensamblaje de orden $n$ si:
\begin{itemize}

\item $\Z\C$ es un anillo abeliano con unidad $C_0$.

\item $N_{ij}^k > 0 \leftrightarrow C_j = C_i^*$.

\item $*$ es un automorfismo sobre $\C$.

\item Se cumple la ecuación característica:

$$ C_iC_j = \sum_{k=0}^n N_{ij}^kC_k $$

\item {\it Conservación de la masa:}

$$ m(C_i)m(C_j) = \sum_{k=0}^n N_{ij}^kC_k $$

donde $m(C_i) = \displaystyle\sum_{j=0}^n N_{ij}^0$.

\end{itemize}

Nos referiremos con $\C$ a la terna $(\C,N,*)$ siempre que no cause confusión.\\

LLamamos {\bf carácter masa } a la aplicación:

$$ \definedfunction{ m }{ \C }{ \N }{ C_i }{ m(C_i) = \displaystyle\sum_{j=0}^n N_{ij}^0 }$$

}
\newpage
Antes de adentrarnos en el teorema principal, nos quedan unas cuantas definiciones más por considerar:

{\bf Masa Total}

Sea:
\begin{itemize}

\item $(\C,N,*)$ un ensamblaje de orden $n$.

\end{itemize}

LLamamos masa total del ensamblaje $(\C,N,*)$ al natural:
\begin{itemize}
\item $\displaystyle\sum_{i=0}^n m(C_i)$
\end{itemize}

y lo denotamos por $m(\C)$.

{\bf Ensamblaje Hermitiano}

Sea:
\begin{itemize}

\item $(\C,N,*)$ un ensamblaje de orden $n$.

\end{itemize}

Entonces, $\C$ es un ensamblaje hermitiano si:
\begin{itemize}

\item $\forall i \in \{0,...,n\}: C_i = C_i^*$.

\end{itemize}

{\bf Ensamblaje Ciclotómico}

Sea:
\begin{itemize}

\item $(\C,N,*)$ un ensamblaje de orden $n$.
\item $ \sigma : \{0,..,n\} \longrightarrow \{0,...,n\} $ una permutación definida por $\sigma(0) = 0, \sigma(n) = 1, \sigma(i) = i+1$.

\end{itemize}

Entonces, $\C$ es un ensamblaje ciclotómico si:
\begin{itemize}

\item $\forall i,j,k \in \{0,...,n\}:\hspace{10pt} N_{\sigma(i)\sigma(j)}^{\sigma(k)} = N_{ij}^k$.

\end{itemize}

{\bf Ensamblajes isomorfos }

Sea:
\begin{itemize}

\item $(\C_1,N_1,*_1),(\C_2,N_2,*_2)$ ensamblajes de orden $n$.

\end{itemize}

Entonces, $\C_1,\C_2$ son isomorfos si $\exists \; \phi : \C_1 \longrightarrow \C_2$ cumpliendo:
\begin{itemize}

\item $\phi(C_iC_j) = \phi(C_i)\phi(C_j)$.

\item $\phi(C_i^*) = \phi(C_i)^*$

\end{itemize}


\section{ Teorema Principal }


Retomemos los periodos Gaussianos. Sea $\C = \{\eta_0 = 1, \eta_1,...,\eta_n\}$. Definamos $N_{ij}^k$ como las apariciones de un elemento de $C_k$ como suma de elementos de $C_i$ y $C_j$ y $*$ como la aplicación que asigna a cada periodo gaussiano su opuesto respecto la suma. Resulta que la terna $(\C,N,*)$ es un ensamblaje ciclotómico de orden $n$ y masa total $p$.

El teorema principal asegura que, salvo isomorfismos, éste es el único ensamblaje que cumple estas condiciones:

{\bf Teorema Principal }

Sea:
\begin{itemize}

\item $p$ un primo.

\item $n$ un natural divisor de $p-1$.

\end{itemize}

Entonces, se cumple salvo isomorfismos:
\begin{itemize}
\item $\exists ! \; \C$ ensamblaje ciclotómico de orden $n$ y masa total $p$
\end{itemize}

Esto permite establecer una correspondencia entre el único subcuerpo de grado $n$ de $\Q(\zeta_p)$ y el único ensamblaje ciclotómico de orden $n$ y masa total $p$.

Así, el estudio de periodos gaussianos se traslada al álgebra abstracta, ofreciendo la capacidad de  analizar las propiedades de los periodos Gaussianos mediante los resultados obtenidos para los ensamblajes ciclotómicos.

La demostración de este teorema, así como el desarrollo de todos los conceptos colaterales que aparecen por el camino se encuentra en el artículo previamente citado:

N.J. Wildberger: {\it On the algebraic structure of Gaussian periods}. University of South Wales, 2003. \href{http://web.maths.unsw.edu.au/~norman/papers/GaussianPeriods.pdf}{link}
\end{document}


